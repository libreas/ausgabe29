\documentclass[a4paper,
fontsize=11pt,
%headings=small,
oneside,
numbers=noperiodatend,
parskip=half-,
bibliography=totoc,
final
]{scrartcl}

\usepackage{synttree}
\usepackage{graphicx}
\setkeys{Gin}{width=.4\textwidth} %default pics size

\graphicspath{{./plots/}}
\usepackage[ngerman]{babel}
\usepackage[T1]{fontenc}
%\usepackage{amsmath}
\usepackage[utf8x]{inputenc}
\usepackage [hyphens]{url}
\usepackage{booktabs} 
\usepackage[left=2.4cm,right=2.4cm,top=2.3cm,bottom=2cm,headheight=25.60228pt,includeheadfoot]{geometry}
\usepackage{eurosym}
\usepackage{multirow}
\usepackage[ngerman]{varioref}
\setcapindent{1em}
\renewcommand{\labelitemi}{--}
\usepackage{paralist}
\usepackage{pdfpages}
\usepackage{lscape}
\usepackage{float}
\usepackage{acronym}
\usepackage{eurosym}
\usepackage[babel]{csquotes}
\usepackage{longtable,lscape}
\usepackage{mathpazo}
\usepackage[flushmargin,ragged]{footmisc} % left align footnote

%%url brekas grrr
\def\UrlBreaks{\do\a\do\b\do\c\do\d\do\e\do\f\do\g\do\h\do\i\do\j\do\k\do\l%
\do\m\do\n\do\o\do\p\do\q\do\r\do\s\do\t\do\u\do\v\do\w\do\x\do\y\do\z\do\0%
\do\1\do\2\do\3\do\4\do\5\do\6\do\7\do\8\do\9\do\-}%

\usepackage{listings}

\urlstyle{same}  % don't use monospace font for urls

\usepackage[fleqn]{amsmath}

%adjust fontsize for part

%% geometry
\clubpenalty = 10000 
\widowpenalty = 10000 
\displaywidowpenalty = 10000
%% tightlist

\providecommand{\tightlist}{%
  \setlength{\itemsep}{0pt}\setlength{\parskip}{0pt}}

\usepackage{sectsty}
\partfont{\large}

%Das BibTeX-Zeichen mit \BibTeX setzen:
\def\symbol#1{\char #1\relax}
\def\bsl{{\tt\symbol{'134}}}
\def\BibTeX{{\rm B\kern-.05em{\sc i\kern-.025em b}\kern-.08em
    T\kern-.1667em\lower.7ex\hbox{E}\kern-.125emX}}

\usepackage{fancyhdr}
\fancyhf{}
\pagestyle{fancyplain}
\fancyhead[R]{\thepage}

%meta

%meta

\fancyhead[L]{N. Jahn \\ %author
LIBREAS. Library Ideas, 29 (2016). % journal, issue, volume.
\href{http://nbn-resolving.de/urn:nbn:de:kobv:11-100238209
}{urn:nbn:de:kobv:11-100238209}} % urn
\fancyhead[R]{\thepage} %page number
\fancyfoot[L] {\textit{Creative Commons BY 3.0}} %licence
\fancyfoot[R] {\textit{ISSN: 1860-7950}}

\title{\LARGE{Rezension zu: Pomerantz, Jeffrey. Metadata. The MIT Press Essential Knowledge Series. The MIT Press, 2015. ISBN: 978-0-262-52851-1. \$15,95}} %title %title
\author{Najko Jahn} %author

\setcounter{page}{132}

\usepackage[colorlinks, linkcolor=black,citecolor=black, urlcolor=blue,
breaklinks= true]{hyperref}

\date{}
\begin{document}

\maketitle
\thispagestyle{fancyplain} 

%abstracts

%body
Die handliche Reihe \enquote{The MIT PRESS Essential Knowledge Series}
widmet eine ihrer neuesten Einführungen dem Thema Metadaten. Jeffrey
Pomerantz, als Informationswissenschaftler zuletzt an der School of
Information and Library Science der University of North Carolina in
Chapel Hill tätig, stellt in acht übersichtlichen Kapiteln die Frage
nach Gegenstand und Zweck von Metadaten. Dabei beschränkt er sich in
seinem Buch, das aus einem Massive Open Online Course (MOOC) heraus
entstanden ist, nicht nur auf die Erschließung wissenschaftlicher
Veröffentlichungen oder die des Kulturerbes. Vielmehr, so seine These,
seien Metadaten aufgrund des Mengenwachstums digitaler Ressourcen sowie
ihrer Vielfalt für deren Auffindbarkeit und Nachnutzung relevanter denn
je zuvor.

Zu Beginn grenzt Pomerantz Metadaten von den Begriffen Information und
Daten ab. Er definiert Metadaten als Aussagen über möglicherweise
informative Gegenstände (\enquote{Metadata is a statement about
potentially informative objects}). Mit dem Bezug auf Aussagen, welche
aus Subjekt, Prädikat und Objekt bestehen, möchte er sich einen Freiraum
verschaffen, um im Verlaufe des Buches auf die Beschreibungsstandards
Dublin Core und das Resource Description Framework (RDF) gesondert
einzugehen. Zu den weiteren Verfahren der Erschließung, in die Pomerantz
zu Beginn einführt und die er in Form eines Glossars übersichtlich
zusammenfasst, gehören Normdateien, Thesauri und Ontologien sowie die
freie Stichwortvergabe mittels Tags.

Nach der Definition der Grundbegriffe und Verfahren der Erschließung
unterscheidet die Einführung zwischen deskriptiven, administrativen und
Nutzungsmetadaten. Die Einführung stellt Dublin Core als Beispiel für
einen deskriptiven Metadatenstandard vor. Erdacht als kleinster
gemeinsamer Nenner während eines Workshops in Dublin (Ohio), dem
Heimatort des Bibliotheksdienstleisters OCLC, umfasst Dublin Core
fünfzehn Elemente zur Beschreibung von Ressourcen im Web. Zwar sei laut
Pomerantz Dublin Core für die Websuche heutzutage kaum mehr relevant.
Allerdings zeigt er, dass sich gegenwärtige Standards in ihrer Arbeit
weiterhin auf Dublin Core und dessen Gestaltungsprinzipien wie dem
universellen Anspruch und die technisch einfache Implementierung
beziehen.

Administrative Metadaten sind laut Pomerantz Metadaten, die den Umgang
mit einer Ressource dokumentieren. Sie umfassen Aussagen zu Technik,
Bewahrung oder den rechtlichen Rahmenbedingungen, unter denen eine
Ressource verwendet werden darf. Mit der Provenienz führt das Buch in
einen wichtigen Aspekt für die Bewertung der Vertrauenswürdigkeit einer
Ressource ein. Wer hat die Ressource erstellt, welche Personen,
Institutionen oder technischen Maßnahmen waren ebenfalls beteiligt oder
wo ist die Ressource erschienen, sind Fragen, die insbesondere in den
Wissenschaften zur Bewertung einer Quelle relevant sind.
Provenienzaussagen können sich hierbei auf eine Sammlung, ein Werk oder
auch auf einzelne Messwerte in einem experimentellen Datensatz beziehen.

Während Pomerantz deskriptive und administrative Metadaten überzeugend
einführt, ist das Kapitel zu Nutzungsmetadaten (\enquote{Use metadata})
weniger gelungen. Zwar weist er darauf hin, dass Nutzungsereignisse in
verschiedenen Bereichen wie dem Handel, der Sicherheit oder Bildung für
analytische Zwecke abgeschöpft werden. Allerdings mangelt es, wie
Pomerantz eingesteht, an terminologischer Klarheit, was unter
\enquote{Nutzungsmetadaten} zu verstehen sei. Möglicherweise wäre hier
der Rückgriff auf die \enquote{Altmetrics} hilfreich gewesen, welche
allerdings keine Erwähnung im Buch findet.

In den nachfolgenden Kapiteln stellt Pomerantz kurz und
allgemeinverständlich die technischen Grundlagen für die Erstellung von
Metadaten vor. Ein besonderes Augenmerk legt er hierbei auf das Semantic
Web, und zwar insbesondere auf die Frage, wie Gedächtnisinstitutionen
ihre Nachweise und Erschließungsverfahren so erweitert haben, damit ihre
Werke im Web auffindbar sind und sich mit anderen Quellen wie der
DBpedia vernetzen lassen. Er illustriert zudem, wie der Webstandard
Schema.org zur strukturierten Suche von Ressourcen über Suchmaschinen
wie Google oder Bing beiträgt.

Die Einführung schließt mit einen Ausblick: Domänenspezifische
Metadaten, die sich an Standards mit einen allgemeinen Anspruch
orientieren wie das Resource Description Framework, werden zur
Nachnutzung vielfältiger digitaler Ressourcen ebenso beitragen wie der
mittlerweile obligatorische Zugriff auf diese Metadaten mittels
Webschnittstellen. Ein Gebiet, auf das Pomerantz in diesem Zusammenhang
gesondert eingeht, ist die eScience. Hier sieht er einen besonderen
Bedarf an Metadaten, die den Erstellungsverlauf beschreiben, um
Vertrauen in die Provenienz einer wissenschaftlichen Ressource zu
schaffen. Kritisch sieht Pomerantz zum Abschluss die Aufzeichnung und
Weitergabe von Metadaten, die die private Kommunikation betreffen. Ohne
etwa den Inhalt von Telefongesprächen aufzuzeichnen, ließen sich über
Metadaten zum Gesprächsverlauf Rückschlüsse auf persönliche Präferenzen
ziehen. Aus diesem Grunde würden seines Erachtens Metadaten in Zukunft
verstärkt an gesellschaftlicher Relevanz gewinnen und Gegenstand
rechtlicher und politischer Auseinandersetzungen.

Die abschließende Betonung normativer Aspekte im Umgang mit Metadaten
weist auf ein Desiderat der Einführung hin: Es fehlt ein eigenständiges
Kapitel über die Rolle von Werten im Erschließungs- und
Bewahrungsprozess. Entscheidungen darüber, wie Bücher oder
wissenschaftliche Datensätze erschlossen werden, können maßgeblichen
Einfluss auf deren Auffindbarkeit und Rezeption haben. Sowohl die
Bibliotheks- und Informationswissenschaft als auch gesellschaftliche
Initiativen wie die \enquote{Radical Reference Collectives} in den USA
problematisieren, dass Metadaten trotz ihres objektiven Anspruchs nicht
davor gefeit sind, Annahmen oder befangene Sichtweisen wiederzugeben,
welche Erkenntnisinteressen oder gesellschaftliche Gruppen
marginalisieren. Ein Kapitel, das insbesondere die Rolle von sozialen
Werten bei der Erschließung im Informationswesen und der Wissenschaft
aufgreift, wäre daher eine verdienstvolle Ergänzung zur ansonsten
lesenswerten Einführung.

Insgesamt bietet die handliche Einführung leicht verständliche Antworten
darauf, wie Metadaten zur Auffindbarkeit und Nachnutzung von Ressourcen
im Web beitragen. Es ist daher nicht nur Studierenden der Bibliotheks-
und Informationswissenschaft zu empfehlen, die nach einem lesenswerten
Einstieg in die Materie der Erschließung suchen. Vielmehr ist es ein
Verdienst der Einführung, dass sie vielfältige Erschließungsverfahren
vorstellt und hierbei besonders auf aktuelle Webstandards eingeht.
Jeffrey Pomerantz' \enquote{Metadata} ist daher auch den Praktikerinnen
und Praktikern sehr zu empfehlen.

%autor
\begin{center}\rule{0.5\linewidth}{\linethickness}\end{center}

\textbf{Najko Jahn} ist Referent an der Universitätsbibliothek Bielefeld
und Mitglied der LIBREAS-Redaktion.

\end{document}