\documentclass[a4paper,
fontsize=11pt,
%headings=small,
oneside,
numbers=noperiodatend,
parskip=half-,
bibliography=totoc,
final
]{scrartcl}

\usepackage{synttree}
\usepackage{graphicx}
\setkeys{Gin}{width=.4\textwidth} %default pics size

\graphicspath{{./plots/}}
\usepackage[ngerman]{babel}
\usepackage[T1]{fontenc}
%\usepackage{amsmath}
\usepackage[utf8x]{inputenc}
\usepackage [hyphens]{url}
\usepackage{booktabs} 
\usepackage[left=2.4cm,right=2.4cm,top=2.3cm,bottom=2cm,headheight=25.60228pt,includeheadfoot]{geometry}
\usepackage{eurosym}
\usepackage{multirow}
\usepackage[ngerman]{varioref}
\setcapindent{1em}
\renewcommand{\labelitemi}{--}
\usepackage{paralist}
\usepackage{pdfpages}
\usepackage{lscape}
\usepackage{float}
\usepackage{acronym}
\usepackage{eurosym}
\usepackage[babel]{csquotes}
\usepackage{longtable,lscape}
\usepackage{mathpazo}
\usepackage[flushmargin,ragged]{footmisc} % left align footnote

%%url brekas grrr
\def\UrlBreaks{\do\a\do\b\do\c\do\d\do\e\do\f\do\g\do\h\do\i\do\j\do\k\do\l%
\do\m\do\n\do\o\do\p\do\q\do\r\do\s\do\t\do\u\do\v\do\w\do\x\do\y\do\z\do\0%
\do\1\do\2\do\3\do\4\do\5\do\6\do\7\do\8\do\9\do\-}%

\usepackage{listings}

\urlstyle{same}  % don't use monospace font for urls

\usepackage[fleqn]{amsmath}

%adjust fontsize for part

%% geometry
\clubpenalty = 10000 
\widowpenalty = 10000 
\displaywidowpenalty = 10000
%% tightlist

\providecommand{\tightlist}{%
  \setlength{\itemsep}{0pt}\setlength{\parskip}{0pt}}

\usepackage{sectsty}
\partfont{\large}

%Das BibTeX-Zeichen mit \BibTeX setzen:
\def\symbol#1{\char #1\relax}
\def\bsl{{\tt\symbol{'134}}}
\def\BibTeX{{\rm B\kern-.05em{\sc i\kern-.025em b}\kern-.08em
    T\kern-.1667em\lower.7ex\hbox{E}\kern-.125emX}}

\usepackage{fancyhdr}
\fancyhf{}
\pagestyle{fancyplain}
\fancyhead[R]{\thepage}

%meta

%meta

\fancyhead[L]{J. Voß \\ %author
LIBREAS. Library Ideas, 29 (2016). % journal, issue, volume.
\href{http://nbn-resolving.de/urn:nbn:de:kobv:11-100238121
}{urn:nbn:de:kobv:11-100238121}} % urn
\fancyhead[R]{\thepage} %page number
\fancyfoot[L] {\textit{Creative Commons BY 3.0}} %licence
\fancyfoot[R] {\textit{ISSN: 1860-7950}}

\title{\LARGE{Wikidata als Universalbibliographie: ein Kommentar}} %title %title
\author{Jakob Voß} %author

\setcounter{page}{}

\usepackage[colorlinks, linkcolor=black,citecolor=black, urlcolor=blue,
breaklinks= true]{hyperref}

\date{}
\begin{document}

\maketitle
\thispagestyle{fancyplain} 

%abstracts
\begin{abstract}
Der Kommentar weist auf die Bedeutung von Wikidata für die Entwicklung
einer Universalbibliographie hin
\end{abstract}

%body
\section*{Ein gescheiterter
Fachartikel}\label{ein-gescheiterter-fachartikel}

Als ich den Call for Papers für die LIBREAS\footnote{\url{http://www.wikidata.org/entity/Q1798120}
  Die Hyperlinks im Text verweisen auf entsprechende Wikidata-Items oder
  -Projektseiten.}-Ausgabe \#29 \enquote{Bibliographien} zur Kenntnis
nahm, war sofort klar: ich muss unbedingt etwas zu einem Thema
einreichen, dass mir schon länger auf den Nägeln brennt. Wie
Wikipedia\footnote{\url{http://www.wikidata.org/entity/Q52}}, deren
Aufstieg ich vor rund zehn Jahren intensiv mit Vorträgen, Publikationen
und nicht zuletzt mit aktiver Teilnahme begleitete, geht es wieder um
ein Projekt der Wikimedia Foundation\footnote{\url{http://www.wikidata.org/entity/Q180}},
und zwar um Wikidata\footnote{\url{http://www.wikidata.org/entity/Q2013}}.
Wieder habe ich das Gefühl, dass hier eine Entwicklung stattfinden, die
eine der Kernaufgabe von Bibliotheken tangiert, aber in ihrer Tragweite
von der Bibliotheks- und Informationswissenschaftlichen\footnote{\url{http://www.wikidata.org/entity/Q13420675}}
Fachcommunity noch nicht so richtig wahrgenommen wird. Wieder versuche
ich zum Thema zu publizieren und nehme an der Entwicklung regen Anteil.
Im Gegensatz zu vor zehn Jahren ist mein Anteil am Wikidata-Projekt
jedoch durch berufliche und familiäre Verpflichtungen beschränkt und
vielleicht fehlt mir auch etwas die naive Unverfrorenheit, die so
hilfreich ist um andere für ein Projekt wie Wikipedia oder Wikidata zu
begeistern.

Inzwischen habe ich den mittlerweile dritten Entwurf eines
Fachartikels\footnote{\url{http://www.wikidata.org/entity/Q591041}} mit
dem Titel \enquote{Wikidata als Universalbibliographie} abgebrochen,
während die Ideen, Notizen, und offene Browsertabs zum Thema immer mehr
werden. Falls jemand mit dem Gedanken einer Masterarbeit, Promotion oder
eines Forschungsprojektes zum Thema spielt, möge sie oder er sich bitte
bei mir melden. Ein Grundproblem besteht in der Geschwindigkeit mit der
sich Wikidata im Allgemeinen und die Sammlung von bibliographischen
Daten in/mit Wikidata im Speziellen entwickelt.

\section*{Wikidata als
Universalbibliographie}\label{wikidata-als-universalbibliographie}

Um es kurz zu machen, möchte ich an dieser Stelle nur meine Einschätzung
abgeben: \emph{Wikidata hat das Potential für Bibliographien
vergleichbares zu leisten wie Wikipedia für allgemeine
Nachschlagewerke}. Mit allen Vor- und Nachteilen.\footnote{Diese Vor-
  und Nachteile wären unter Anderem Gegenstand einer genaueren
  Untersuchung der tatsächlichen und prognostizierten Rolle von Wikidata
  für Bibliographien.}\footnote{Für den Bereich der Normdaten ist
  Wikidata übrigens von ähnlicher Bedeutung, dies ist aber ein anderes
  Thema (vgl. Voß u.~a.).} Insbesondere ist seit Paul Otlets Répertoire
bibliographique universel\footnote{\url{http://www.wikidata.org/entity/Q3456262}}
das Ziel einer wirklichen Universalbibliographie erstmals wieder in
greifbare Nähe gerückt.\footnote{Siehe Hartmann (2015) für eine
  Auseinandersetzung mit Otlets Projekt in LIBREAS, aus der bereits der
  Datenbank-Charakter dieser Universalbibliographie hervorgeht.} Diese
Einschätzung basiert nicht nur auf meiner allgemeinen Erfahrung mit
Wikipedia, Wikidata, Social Cataloging, bibliographischen Datenformaten
und Datenbanken etc., sondern kann auch durch einige Indizien belegt
werden:

\begin{itemize}
\tightlist
\item
  Dank ihres flexiblen Datenmodells lassen sich bereits jetzt
  bibliographische Daten in Wikidata eintragen. Zur Koordinierung gibt
  es verschiedene Projekte innerhalb von Wikidata (WikiProject
  Books\footnote{\url{https://www.wikidata.org/wiki/Wikidata:WikiProject_Books}},
  WikiProject Periodicals\footnote{\url{https://www.wikidata.org/wiki/Wikidata:WikiProject_Periodicals}},
  WikiProject Source MetaData\footnote{\url{https://www.wikidata.org/wiki/Wikidata:WikiProject_Source_MetaData}},
  \ldots{}). Wie in Wikipedia gibt es allerdings keine zentrale
  Koordination, so dass von verschiedener Seite bibliographische Angaben
  in Wikidata einfließen.\footnote{Ein einfaches Beispiel ist die von
    Willighagen (2016) beschriebene Migration seiner Datenbank zu
    Wikidata: Zwischen 2010 und 2016 sammelten Egon Willighagen und
    Samuel Lampa die Säurekonstante (\(pK_a\)) verschiedener chemischer
    Substanzen in einer Datenbank. Jeder Eintrag besteht aus dem
    International Chemical Identifier (InChI), einem Messwert und einer
    Fachpublikation in welcher der Messwert publiziert wurde. In
    Wikidata würden im Rahmen des Umzugs der Datenbank nach Wikidata für
    alle Fachpublikationen einzelne Wikidata-Einträge angelegt.}
\end{itemize}

\begin{itemize}
\item
  Während beim Social Cataloging herkömmlicherweise jedeR NutzerIn eine
  eigene Bibliographie pflegt (\enquote{bag-model}) arbeiten in Wikidata
  alle gemeinsam an einem Datenbestand (\enquote{set-model}), ähnlich
  wie bei einem Verbundkatalog.
\item
  Verglichen mit bibliothekarischer Verbundkatalogisierung ist die Hürde
  Fehler zu beseitigen oder Ergänzungen vorzunehmen in Wikidata
  allerdings ungleich niedriger. Während die Mittel von Bibliotheken
  eher begrenzt sind, ist bei Wikidata von einer weiter wachsenden Zahl
  von Beitragenden auszugehen.
\item
  Die umfangreiche Verfügbarkeit der Inhalte von Wikidata über
  verschiedene Schnittstellen ermöglicht es, qualifizierte Aussagen über
  die Datenqualität zu treffen und diese so kontrolliert zu verbessern.
\item
  Als universelle Datenbank ist Wikidata nicht auf bibliographische
  Daten beschränkt. Das Prinzip der Verknüpfung mit Normdaten\footnote{\url{http://www.wikidata.org/entity/Q6423319}}
  lässt sich so auf die Spitze treiben und ermöglicht bibliometrische
  und weitere Auswertungen, die mit anderen Katalogen nur schwer möglich
  sind.
\item
  Wikidata ist weder kommerziellen noch politischen Interessen
  unterworfen, die die Entwicklung von (Universal)bibliographien in
  anderen Bereichen behindern.
\item
  Die Verwendung von Wikidata für die Wissenschaft wird auch in anderen
  Bereichen vorangetrieben. Der Antrag zum EU-Projekt Wikidata for
  Research (Wiki4R) gibt einen guten Überblick über die zu erwartende
  Entwicklung (Mietchen u.~a., 2015).
\item
  Ende Mai findet in Berlin die WikiCite-Tagung\footnote{\url{https://meta.wikimedia.org/wiki/WikiCite_2016}}
  mit 50 ExpertInnen statt, um einen konkreten Plan für die Umsetzung
  der Migration aller Quellenangaben aus Wikipedia nach Wikidata
  festzulegen.
\end{itemize}

Es kann also davon ausgegangen werden, dass alle die sich mit der
Sammlung bibliographischer Daten beschäftigen mit Wikidata \enquote{in
interessanten Zeiten leben}\footnote{\url{http://www.wikidata.org/entity/Q14634108}}
werden.

\hyperdef{}{references}{\label{references}}
\section*{Literaturangaben}\label{literaturangaben}
\addcontentsline{toc}{section}{Literaturangaben}

\hyperdef{}{ref-Hartmann2015}{\label{ref-Hartmann2015}}
Hartmann, F. (2015). Paul Otlets Hypermedium. Dokumentation als
Gegenidee zur Bibliothek. \emph{LIBREAS. Library Ideas}. URL:
\url{http://libreas.eu/ausgabe28/04hartmann/}.

\hyperdef{}{ref-Wiki4R}{\label{ref-Wiki4R}}
Mietchen, D., Hagedorn, G., Willighagen, E., Rico, M., Gómez-Pérez, A.,
Aibar, E., Rafes, K., Germain, C., Dunning, A., Pintscher, L., \&
Kinzler, D. (2015). Enabling Open Science: Wikidata for Research
(Wiki4R). \emph{Research Ideas and Outcomes} 1, e7573.
\url{http://doi.org/10.3897/rio.1.e7573}.

\hyperdef{}{ref-Voss2014}{\label{ref-Voss2014}}
Voß, J., Bausch, S., Schmitt, J., Bogner, J., Berkelmann, V., Ludemann,
F., Löffel, O., Kitroschat, J., Bartoshevska, M., \& Seljuzki, K.
\emph{Normdaten in Wikidata}. lulu.com. URL:
\url{https://hshdb.github.io/normdaten-in-wikidata/}.

\hyperdef{}{ref-Willighagen2016}{\label{ref-Willighagen2016}}
Willighagen, E. (2016). Migrating pKa data from DrugMet to Wikidata.
\emph{chem-bla-ics}. URL:
\url{http://chem-bla-ics.blogspot.de/2016/03/migrating-pka-data-from-drugmet-to.html}.

%autor
\begin{center}\rule{0.5\linewidth}{\linethickness}\end{center}

\textbf{Jakob Voß} arbeitet im Bereich Forschung und Entwicklung an der
Verbundzentrale des GBV (VZG).

\end{document}