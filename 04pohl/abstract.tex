Am Hochschulbibliothekszentrum des Landes Nordrhein-Westfalen (hbz) wird
seit Anfang 2014 nach Vorgaben und unter Begutachtung der Universitäts-
und Landesbibliotheken in Düsseldorf, Münster und Bonn ein neuer
Webauftritt für die Landesbibliographie Nordrhein-Westfalens, die
Nordrhein-Westfälische Bibliographie (NWBib) entwickelt. Die Entwicklung
basiert auf der Web- Schnittstelle des Linked-Open-Data-Dienst lobid und
wird vollständig mit Open- Source-Software entwickelt. Aus der
Perspektive des Entwicklungsteams am hbz beschreibt der Artikel Kontext
und Durchführung des Projekts. Der Beitrag skizziert die historische
Entwicklung der NWBib mit Fokus auf die Beziehung der Bibliographie zum
World Wide Web (WWW), erläutert die Voraussetzungen für die
Neuentwicklung sowie die Leitlinien des Entwicklungsprozesses, gibt
einen Überblick über die Nutzung des neuen Webauftritts und die zur
Umsetzung verwendete Technologie. Abgeschlossen wir der Artikel mit
Lessons-Learned und einem Ausblick auf weitere Entwicklungen.
