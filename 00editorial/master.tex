\documentclass[a4paper,
fontsize=11pt,
%headings=small,
oneside,
numbers=noperiodatend,
parskip=half-,
bibliography=totoc,
final
]{scrartcl}

\usepackage{synttree}
\usepackage{graphicx}
\setkeys{Gin}{width=.4\textwidth} %default pics size

\graphicspath{{./plots/}}
\usepackage[ngerman]{babel}
\usepackage[T1]{fontenc}
%\usepackage{amsmath}
\usepackage[utf8x]{inputenc}
\usepackage [hyphens]{url}
\usepackage{booktabs} 
\usepackage[left=2.4cm,right=2.4cm,top=2.3cm,bottom=2cm,headheight=25.60228pt,includeheadfoot]{geometry}
\usepackage{eurosym}
\usepackage{multirow}
\usepackage[ngerman]{varioref}
\setcapindent{1em}
\renewcommand{\labelitemi}{--}
\usepackage{paralist}
\usepackage{pdfpages}
\usepackage{lscape}
\usepackage{float}
\usepackage{acronym}
\usepackage{eurosym}
\usepackage[babel]{csquotes}
\usepackage{longtable,lscape}
\usepackage{mathpazo}
\usepackage[flushmargin,ragged]{footmisc} % left align footnote

%%url brekas grrr
\def\UrlBreaks{\do\a\do\b\do\c\do\d\do\e\do\f\do\g\do\h\do\i\do\j\do\k\do\l%
\do\m\do\n\do\o\do\p\do\q\do\r\do\s\do\t\do\u\do\v\do\w\do\x\do\y\do\z\do\0%
\do\1\do\2\do\3\do\4\do\5\do\6\do\7\do\8\do\9\do\-}%

\usepackage{listings}

\urlstyle{same}  % don't use monospace font for urls

\usepackage[fleqn]{amsmath}

%adjust fontsize for part

%% geometry
\clubpenalty = 10000 
\widowpenalty = 10000 
\displaywidowpenalty = 10000
%% tightlist

\providecommand{\tightlist}{%
  \setlength{\itemsep}{0pt}\setlength{\parskip}{0pt}}

\usepackage{sectsty}
\partfont{\large}

%Das BibTeX-Zeichen mit \BibTeX setzen:
\def\symbol#1{\char #1\relax}
\def\bsl{{\tt\symbol{'134}}}
\def\BibTeX{{\rm B\kern-.05em{\sc i\kern-.025em b}\kern-.08em
    T\kern-.1667em\lower.7ex\hbox{E}\kern-.125emX}}

\usepackage{fancyhdr}
\fancyhf{}
\pagestyle{fancyplain}
\fancyhead[R]{\thepage}

%meta

%meta

\fancyhead[L]{Redaktion LIBREAS \\ %author
LIBREAS. Library Ideas, 29 (2016). % journal, issue, volume.
\href{http://nbn-resolving.de/urn:nbn:de:kobv:11-100238106
}{urn:nbn:de:kobv:11-100238106}} % urn
\fancyhead[R]{\thepage} %page number
\fancyfoot[L] {\textit{Creative Commons BY 3.0}} %licence
\fancyfoot[R] {\textit{ISSN: 1860-7950}}

\title{\LARGE{Editorial \#29: Bibliographien}} %title %title
\author{Redaktion LIBREAS} %author

\setcounter{page}{1}

\usepackage[colorlinks, linkcolor=black,citecolor=black, urlcolor=blue,
breaklinks= true]{hyperref}

\date{}
\begin{document}

\maketitle
\thispagestyle{fancyplain} 

%abstracts

%body
Die Bibliographie und das Bibliographieren, so eine These im Call for
Papers für diese Ausgabe, sind originär bibliothekarische Themen --nach
dem Durcharbeiten der Einreichungen können wir feststellen: Sie trifft
gleichzeitig zu und trifft nicht zu. Zum einen ist der Begriff des
Bibliographierens nicht geschützt, schon gar nicht historisch, und damit
ist er interpretierbar. Zwei Texte dieser Ausgabe gehen auf eine
Bibliographie ein, die für die Geschichte der Homosexuellenbewegung
wichtig war, obwohl diese Bibliographie selber im Rahmen
bibliothekarischer Diskussionen vielleicht nicht als solche benannt
würde. Zudem ist das Bibliographieren heute offenbar nicht ohne die
Möglichkeiten der Wikimedia zu denken. Erstaunlich ist vielleicht auch,
dass in den Texten technische Fragen einen großen Stellenwert einnehmen,
nicht inhaltliche. Eine weitere These des Call for Papers war, dass die
Bibliographie und auch die Arbeit des Bibliographierens
untertheoretisiert ist. Die Überraschung, dass uns vor allem Texte zu
technischen Fragestellungen erreichten, mag mit dieser geringen
Theoretisierung zusammenhängen. Wir hätten uns mehr theoretische
Auseinandersetzungen gewünscht, die jetzt nur mit einem sehr
verspielten, anregenden Text zu Metabibliographien vertreten sind.

Dies soll nicht als Klage gelten: die LIBREAS. Library Ideas versteht
sich als ein Ort im bibliothekarischen und bibliothekswissenschaftlichen
Diskurs, und da dieser Diskurs so technisch orientiert ist, spiegelt
sich dies auch in der Ausgabe wieder. In einem frühen Stadium der
Entwicklung des Schwerpunkts dieser Ausgabe gab es auch die Idee, vor
allem nach der Zukunft der Bibliographie zu fragen. Wir sind davon
abgerückt (obwohl die Frage weiter Teil des Call for Papers blieb, weil
sie wichtig ist), da viel zu oft im bibliothekarischen Diskurs nach der
Zukunft gefragt wird, ohne die gegenwärtige Situation zu klären,
obgleich Aussagen über die Zukunft ohne einer Verankerung in der
gegenwärtigen Situation schnell den Fokus und die Bodenhaftung verlieren
können. Allerdings ist jetzt, bei der Herausgabe der aktuellen Ausgabe,
auch zu erkennen, dass wir durch diese Entscheidung vielleicht zu wenig
gerade über diese Themen erfahren haben. Dabei ist es etwas, was die
Zunft der Bibliographierenden aktuell beschäftigt. Es bleibt für andere
Ausgaben und vielleicht auch andere Publikationsorte offen. Davon
abgesehen hoffen wir, die LIBREAS \#29 ist für Sie/euch als Lesende so
interessant wie für uns als Redaktion -- und hoffentlich auch eine
Anregung, verstärkt nicht nur Themen aus einer
bibliothekswissenschaftlichen Perspektive, sondern auch spezifisch
bibliothekarische Themen zu behandeln.

Ihre/eure Redaktion LIBREAS. Library Ideas

(Berlin, Chur, Dresden, Exeter, München)

%autor

\end{document}